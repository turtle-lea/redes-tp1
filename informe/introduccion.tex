\emph{Adress resolution protocol} permite mapear direcciones IP a direcciones físicas. Cuando un host desea comunicarse con otro
dentro de una misma red local necesita conocer la \emph{MAC} adress del mismo. Conociendo su dirección IP, el protocolo le permite
enviar un paquete \emph{who-has} mediante un broadcast para obtener dicha dirección. De esta manera recibe un paquete unicast
\emph{ARP-reply} por parte del host destino que tiene dicha dirección IP.
Para evitar una cantidad excesiva de broadcasts, cada nodo de la red mantiene su propia tabla de mapeos que actualiza luego de cierto tiempo.

Cada paquete incluye numerosos campos (Hardware type, protocol type, etc). Sin embargo, pondremos énfasis especialmente en los siguientes:
\begin{itemize}
    \item Operation: 1 == request, 2 == reply
    \item Sender hardware adress: \emph{MAC} address source
    \item Sender protocol adress: \emph{IP} address source
    \item Target hardware adress: \emph{MAC} address dest
    \item Target protocol adress: \emph{IP} address dest
\end{itemize}

El objetivo del trabajo práctico consiste en analizar el flujo de paquetes \emph{who-has} en distintas redes 
no contraladas. De esta manera podemos corroborar de manera empírica el comportamiento establecido por el protocolo \emph{ARP} 
para el envío de paquetes en una red local.  







A partir de los paquetes ARP que capturamos con la función $sniff$ de la librería scapy, graficamos un grafo que relaciona ips como 
nodos de manera tal que exista un eje del nodo ip\_src al nodo ip\_dst si se capturó al menos un paquete \emph{who-has} emitido por 
ip\_src preguntando por la mac\_addr de la ip\_dst.

