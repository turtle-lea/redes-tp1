\section{Introducci\'on}

El protocolo ARP (\emph{Adress Resolution Protocol}) permite mapear direcciones
IP a direcciones físicas. Cuando un host desea comunicarse con otro dentro de
una misma red local debe hacerlo utilizando la \emph{MAC Adress} del mismo.
Conociendo su dirección IP, puede \emph{brodcastear} un paquete 
\emph{ARP who-has} pidiendo la \emph{MAC} asociada a dicha IP. De esta manera
recibirá en respuesta un paquete \emph{ARP-reply} por parte del host destino.
Para evitar una cantidad excesiva de broadcasts, cada nodo de la red mantiene
su propia tabla de mapeos que actualiza peri\'odicamente.

Cada paquete ARP incluye numerosos campos. Sin embargo, pondremos énfasis 
especialmente en los siguientes: Operation (indica si es un pedido o respuesta),
Sender Hardware Adress, Sender Protocol Adress, Target Hardware Adress y 
Target Protocol Adress. 

El objetivo de este trabajo práctico es analizar el flujo de paquetes 
\emph{who-has} en distintas redes p\'ublicas y privadas. Luego, utilizando
como base el concepto de \textit{entrop\'ia de la informaci\'on} analizaremos el
rol que juega cada dispositivo dentro de la red.
