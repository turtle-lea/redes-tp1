El protocolo ARP (\emph{Adress Resolution Protocol}) permite mapear direcciones IP a direcciones
físicas. Cuando un host desea comunicarse con otro dentro de una misma red local debe hacerlo
utilizando la \emph{MAC Adress} del mismo. Conociendo su dirección IP, puede \emph{brodcastear} un
paquete \emph{ARP who-has} pidiendo la \emph{MAC} asociada a dicha IP. De esta manera recibirá en
respuesta un paquete \emph{ARP-reply} por parte del host destino.
Para evitar una cantidad excesiva de broadcasts, cada nodo de la red mantiene su propia tabla de
mapeos que actualiza luego de cierto tiempo.

Cada paquete ARP incluye numerosos campos. Sin embargo, pondremos énfasis especialmente en los 
siguientes:
\begin{itemize}
    \item Operation: 1 (Request), 2 (Reply)
    \item Sender hardware adress: \emph{MAC} address source
    \item Sender protocol adress: \emph{IP} address source
    \item Target hardware adress: \emph{MAC} address dest
    \item Target protocol adress: \emph{IP} address dest
\end{itemize}

El objetivo de este trabajo práctico es analizar el flujo de paquetes \emph{who-has} en distintas
redes no contraladas. Luego, en base a la informaci\'on recopilada, analizaremos el rol que juega
cada dispositivo dentro de la red. De esta manera podemos corroborar de manera empírica el 
comportamiento establecido por el protocolo \emph{ARP} para el envío de paquetes en una red local.  
