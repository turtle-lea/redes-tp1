El primer paso en este trabajo fue implementar un script en \emph{python} para escuchar pasivamente
paquetes dentro de la red local. Para esto capturamos todos los paquetes que circulan utilizando la 
librería Scapy, y nos quedamos solo con los ARP de tipo \emph{who-has}. Luego, contamos cuantas veces
aparece una direcci\'on en cada campo, as\'i como tambi\'en la interacci\'on entre todos los pares
de ip. Terminamos ordenando esta información según la cantidad de paquetes, de mayor a menor.


Con estas observaciones nos proponemos a calcular la entropia tomando dos modelos distintos de fuente:
en \textbf{$S_{src}$} los símbolos son las direcciones IP que aparecen en el campo SRC; en 
\textbf{$S_{dst}$} son las direcciones IP que aparecen en el campo DST.


En relación a esto tenemos una mejor base para realizar un análisis del estado de 
la red al momento de las mediciones, y podemos analizar estadísticamente qué IPs 
son más relevantes o significativas en la LAN utilizando la información del símbolo 
con respecto a la entropía de la fuente correspondiente.


En la próxima sección podrán verse los datos obtenidos y el análisis de los mismos
para las distintas redes en las cuales fueron realizadas las capturas: 
\textcolor{red}{Entrepiso DC} (\textbf{2634} paquetes), \textcolor{red}{Fibertel Zone} de Starbucks (\textbf{500} paquetes),
\textcolor{red}{Casa de Familia} (\textbf{515} paquetes) y \textcolor{red}{Empresa} (\textbf{400} paquetes).



