El primer paso en este trabajo fue implementar una $tool$ para escuchar pasivamente paquetes de una red local dada. 
Para esto utilizamos la librería Scapy de Python, la cual provee una gran variedad
de herramientas para trabajar con distintos paquetes de diversos protocolos.

Básicamente utilizamos la función \emph{sniff} que captura paquetes que se transmiten
por la red en la cual estamos conectados, y se filtran los paquetes que no son ARP, 
o que tienen la operation reply, quedándonos únicamente con los \emph{who-has}.

Luego, comprimimos estos datos para ahorrar extensión, es decir, si una dirección pregunta varias veces por otra,
esto se reduce a una tripla \textbf{n paquetes enviados | ip\_src | ip\_dst}. Terminamos ordenando esta información según la cantidad de paquetes, de mayor a menor.

Con estas observaciones nos proponemos a calcular la entropia de dos tipos
distintos de fuente según el modelo presentado: \textbf{$S_{src}$} y \textbf{$S_{dst}$}


donde los símbolos son las direcciones IP que aparecen en los paquetes ARP who-has
como fuente y como destino respectivamente.

En relación a esto tenemos una mejor base para realizar un análisis del estado de 
la red al momento de las mediciones, y podemos analizar estadísticamente qué IPs 
son más relevantes o significativas en la LAN utilizando la información del símbolo 
con respecto a la entropía de la fuente correspondiente.

En la próxima sección podrán verse los datos obtenidos y el análisis de los mismos
para las distintas redes en las cuales fueron realizadas las capturas: \textcolor{red}{Entrepiso DC} (\textbf{2634} paquetes), \textcolor{red}{Fibertel Zone} de Starbucks (\textbf{500} paquetes)

