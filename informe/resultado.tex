\subsection{Casa Familia}

\begin{wrapfigure}{R}{0.4\textwidth}
\vspace{-35pt}
\hspace{-35pt}
\centering
   \includegraphics[width=0.4\textwidth]{resultados/casa/conectividadNX.pdf}
\vspace{-30pt}
   \caption{Grafo de la red}
\end{wrapfigure}

En el caso de la casa familiar sab\'iamos de antemano que la direcci\'on del router
es 192.168.1.1 y que 192.168.1.35 pertenece al dispositivo que m\'as tiempo estuvo
prendido, el mismo es una notebook. La ip del router es de las que aparece con mayor
frecuencia en los campos SRC o DST, mientras que $.35$ solo aparece un alto n\'umero
de veces en DST. 

Ambos grafos presentan forma de estrella, teniendo al router como centro. 
La entrop\'ia fue: $1.68$ para el modelo $SRC$ y $2.67$ para el modelo DST.
Cabe destacar que aunque no est\'a reflejado en los gr\'aficos, sucedi\'o algo interesante, 
aparecieron paquetes ARP con direcciones que no pertenecian a la red local.

\begin{figure}[H]
	\center
	\begin{subfigure}{0.4\textwidth}
		\includegraphics[width=1.0\textwidth]{resultados/casa/ipsSrc_1_6805902069.pdf}
		\caption{Estimaci\'on de la probabilidad de cada s\'imbolo en modelo SRC}
	\end{subfigure}
	~
	\begin{subfigure}{0.4\textwidth}
		\includegraphics[width=1.0\textwidth]{resultados/casa/ipsDst_2_67355481854.pdf}
		\caption{Estimaci\'on de la probabilidad de cada s\'imbolo en modelo DST}
	\end{subfigure}
\end{figure}


\subsection{Starbucks}

\begin{wrapfigure}{R}{0.4\textwidth}
\vspace{-35pt}
\hspace{-35pt}
\centering
   \includegraphics[width=0.4\textwidth]{resultados/starbucks/conectividadNX.pdf}
\vspace{-30pt}
   \caption{Grafo de la red}
\end{wrapfigure}


Para el caso de la red abierta disponible en Starbucks, podemos ver en el 
grafo simple una notoria centralizaci\'on de paquetes hacia/desde la direcci\'n
10.254.88.1, la cual a su vez posee una alta frecuencia de aparici\'on en ambos
modelos. Esto ser\'ia el comportamiento esperado del router de la red
10.254.88.0/24. Asumimos que \'esta es la direcci\'on de la red puesto que 
la mayor\'ia de las IPs de los paquetes capturados difieren en el \'ultimo octeto.
Por otro lado, podemos ver que la segunda direcci\'on con mayor frecuencia en DST
fue 10.254.80.1, seguida por 10.254.88.8. Suponemos que $88.8$ es la pc del lugar
dada la ip baja, y la cantidad de apariciones. Por otro lado, no sabemos que es
$80.1ç$ dado que parece formar una red aparte de la sniffeada.

La entrop\'ia fue: $4.61$ para el modelo $SRC$ y $3.76$ para el modelo
DST.


Nuevamente entre los paquetes capturados volvieron a aparecer direcciones que
no parecieran pertenecer a la red local, resultando sumamente interesante el caso
de 10.254.80.1

\begin{figure}[H]
	\center
	\begin{subfigure}{0.4\textwidth}
		\includegraphics[width=1.0\textwidth]{resultados/starbucks/ipsSrc_4_6187931499.pdf}
		\caption{Estimaci\'on de la probabilidad de cada s\'imbolo en modelo SRC}
	\end{subfigure}
	~
	\begin{subfigure}{0.4\textwidth}
		\includegraphics[width=1.0\textwidth]{resultados/starbucks/ipsDst_3_76848714287.pdf}
		\caption{Estimaci\'on de la probabilidad de cada s\'imbolo en modelo DST}
	\end{subfigure}
\end{figure}


\subsection{Entrepiso}

En el caso del entrepiso los grafos muestran tres nodos con la forma que caracteriz\'o 
al router dentro de la casa, estos nodos son: 10.1.100.254, 10.1.200.30 y 10.1.200.254.
Sin embargo, estos poseen al menos la mitad de las aparaciones que otros nodos en 
el modelo DST, cosa que $200.30$ repite en SRC.


\subsection{Discusi\'on}


\begin{itemize}
	\item Por un lado, el componente m\'as pequeño, el cual se centraliza 
en la direcci\'on 10.254.80.1. Suponemos que \'esta direcci\'on es la de un router
de otra red (10.254.80.0/24) y en alg\'un momento sniffeamos paquetes de dicha 
red sin enterarnos.
	\item Por otro lado, la direcci\'on 169.254.255.255. Investigamos esta
IP y encontramos que es utilizada como broadcast por DHCP que es un protocolo
de configuraci\'on autom\'atica de par\'ametros de red tales como direcciones IP
para interfaces y servicios.
	\item Por \'ultimo, la direcci\'on 0.0.0.0. \textcolor{red}{FALTA EXPLICAR y poner referencias}
\end{itemize} 

