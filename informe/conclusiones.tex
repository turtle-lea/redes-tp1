Vimos que capturando paquetes who-has de una red se puede hacer cierto análisis
de la misma y determinar de alguna forma la estructura de la red, reconociendo
IPs de mayor interacción a nivel ARP como fue el caso de routers o servidores.

Cuando se analiza una red pequeña, por ejemplo, una LAN hogareña, el $sniffeo$
de estos paquetes permite un analisis quizá más detallado. Los dispositivos 
conectados a la red se comunican principálmente con el router y al no ser 
demasiadas las IPs involucradas, los grafos son pequeños y bastante claros, del
tipo estrella centralizados en el router.

En el caso de otras redes más grandes, la conectividad empieza a enroscarse y 
analisis se complica. Ya los routers no son tan claros de encontrar, e IPs de 
otros dispositivos (servidores) también toman relevancia, quizás en mayor medida
que un router. 
