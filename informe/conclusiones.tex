\section{Conclusiones}

Capturando paquetes who-has de una red es posible detectar los nodos 
relevantes de una red, por ej, en todos los casos los routers resultaron
ser facilmente detectables al representar en forma de grafo los intercambios
de paquetes ARP por parte de pares de nodos, dado que los mismos son quienes
interactuan con la mayor cantidad de nodos. 

Por otra parte, el analizar la frecuencia de aparici\'on de las ips en cada
campo junto con el c\'alculo de la entrop\'ia en la red permiti\'o
demostrar emp\'iricamente que la entrop\'ia es mas baja cuando son pocos los 
s\'imbolos que aparecen
con mucha frecuencia. A su vez, vimos que de acuerdo a como modelaramos la
fuente los resultados eran distintos. Pero no solo los resultados en valores,
en los modelos SRC el router siempre ocup\'o una posici\'on
menor que en el modelo DST. 
Esto puede ser de ayuda para localizar dispositivos importantes en redes
de entrop\'ia baja, por ej, en la casa y la empresa. En redes
mas complejas como la de starbucks, no aport\'o ninguna informaci\'on mas
que demostrar que la red era ca\'otica. 
